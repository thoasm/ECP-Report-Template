
\documentclass{ecpreport-publicv1}
%\usepackage{draftwatermark}        % Use this if you want to have a DRAFT watermark
\usepackage{wrapfig}
\usepackage{subfig}
\usepackage{xcolor,colortbl}
\usepackage{array}
\usepackage{float} %added to wrangle tables and figures
\usepackage[disable]{todonotes}%added todonotes for hardcopy notes and to span repos -STC
\newcolumntype{L}[1]{>{\raggedright\let\newline\\\arraybackslash\hspace{0pt}}m{#1}}
\definecolor{Gray}{gray}{0.85}
\definecolor{LightCyan}{rgb}{0.88,1,1}

\usepackage[T1]{fontenc}
\usepackage{lmodern}
%\usepackage{kpfonts}

\newcolumntype{g}{>{\columncolor{Gray}}c}
\newcolumntype{w}{>{\columncolor{white}}c}
%%\begin{flushright}
%%
%%\end{flushright}
\newcolumntype{y}{>{\columncolor{LightCyan}}c}

%\usepackage{subcaption}

\usepackage[framemethod=TikZ,backgroundcolor=gray!40,shadow=true,roundcorner=8pt]{mdframed}
\usetikzlibrary{shadows}
\usepackage{hyperref}
\hypersetup{
	colorlinks,
	citecolor=blue,
	filecolor=blue,
	linkcolor=blue,
	urlcolor=blue
}
%\usepackage{subcaption}
\usepackage{graphicx}
\usepackage{grffile}
\usepackage{listings}
\usepackage{csquotes}
\lstset{
  basicstyle=\linespread{0.94}\footnotesize\ttfamily,
  columns=fullflexible,
  language=C++,
  escapeinside={(*@}{@*)},
  breaklines=true,
}

\usepackage{tikz}
\usepackage{pgfplots}
\usepackage{paralist}


% Re-do the titleformat to better suite a small report
\titleformat{\section}{\centering\bf\large}{\thetitle.}{1ex}{}[]
\titleformat{\subsection}{\bf}{\thetitle}{1ex}{}[]
\titleformat*{\subsubsection}{\bf\itshape}


%%---------------------------------------------------------------------------%%
%% VARIABLES
%%---------------------------------------------------------------------------%%
\author{Author 1, Organization 1
  \and Author 2, Organization 2
  }
\title{ECP Report Template}
%\date{\today}
\date{February 03, 2022}
%\wbs{2.3}
\id{ECP Report Template ID}
%\reportnum{ECP Report Template Number}
\reportnum{} % Should remain empty to not show above the title


%%---------------------------------------------------------------------------%%
%% FRONT MATTER
%%---------------------------------------------------------------------------%%

\begin{document}
\frontmatter


% Example of a Revision log, but this is rarely used in a Report
%%---------------------------------------------------------------------------%%
%% REVISION LOG

%\begin{revlog}
%
%  1.0 & July 1, 2018 & \textit{ECP ST Capability Assessment Report } \\\hline
%  1.5 & February 1, 2019 & \textit{Second release} \\\hline
%  2.0 & February 1, 2020 & \textit{Third release} \\\hline
%  2.5 & November 19, 2020 & \textit{Fourth release}\\\hline
%  3.0 & November 22, 2021 & \textit{Fifth release--DRAFT} \\\hline
%\end{revlog}

%%---------------------------------------------------------------------------%%

% EXECUTIVE SUMMARY
% Anything in-between \begin{abstract} and \end{abstract} will be under the headline "EXECUTIVE SUMMARY"


% The following can be enabled, but in reports, they are rarely needed
%\tableofcontents
%\listoffigures
%\listoftables

%%---------------------------------------------------------------------------%%
% MAIN DOCUMENT
%%---------------------------------------------------------------------------%%

\mainmatter
\section{Introduction}\label{sect:intro}

Lorem ipsum dolor sit amet, consectetur adipiscing elit, sed do eiusmod tempor incididunt ut labore et dolore magna aliqua. Feugiat scelerisque varius morbi enim nunc faucibus. Ut ornare lectus sit amet est placerat. Enim praesent elementum facilisis leo vel fringilla est ullamcorper eget. In pellentesque massa placerat duis ultricies lacus sed turpis. Integer malesuada nunc vel risus commodo viverra maecenas accumsan lacus. Aliquet eget sit amet tellus cras. Orci dapibus ultrices in iaculis. Feugiat pretium nibh ipsum consequat nisl vel pretium lectus quam. Dui nunc mattis enim ut. Dignissim sodales ut eu sem integer vitae justo. Aliquet eget sit amet tellus cras adipiscing. A diam sollicitudin tempor id eu nisl nunc mi.~\cite{anzt2020ginkgo}



%%---------------------------------------------------------------------------%%
%\vspace{3in}
\clearpage

\newpage
\section*{Acknowledgments}

This research was supported by the Exascale Computing Project (ECP), Project
Number: 17-SC-20-SC, a collaborative effort of two DOE organizations---the
Office of Science and the National Nuclear Security
Administration---responsible for the planning and preparation of a capable
Exascale ecosystem---including software, applications, hardware, advanced
system engineering, and early testbed platforms---to support the nation's
Exascale computing imperative.

%%---------------------------------------------------------------------------%%
%% References
\newpage
\bibliographystyle{unsrt}
\bibliography{references}
\end{document}
